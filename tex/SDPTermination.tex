\documentclass[10pt, a4paper]{article}
\usepackage[utf8]{inputenc}
\usepackage[T1]{fontenc}
\usepackage[american]{babel}
\usepackage[dvipsnames]{xcolor}
\usepackage{amsmath}
\usepackage{amssymb}
\usepackage{amsthm}
\usepackage{dsfont}
\usepackage{mathtools}
\usepackage{graphicx}
\usepackage{enumitem}
\usepackage{booktabs}
\usepackage{siunitx}
\usepackage[lined,ruled,noend,linesnumbered]{algorithm2e}
\usepackage[color=red!50!blue!50,textsize=footnotesize,textwidth=4cm]{todonotes}
\usepackage[breaklinks,colorlinks,citecolor=blue,linkcolor=blue]{hyperref}
% \usepackage[tmargin=2.5cm,bmargin=2.5cm,lmargin=2.8cm,rmargin=2.8cm]{geometry}
\usepackage{cleveref}
\usepackage{xspace}

% operators
\DeclareMathOperator{\conv}{conv}
\DeclareMathOperator{\cone}{cone}
\DeclareMathOperator{\aff}{aff}
\DeclareMathOperator{\lin}{lin}
\DeclareMathOperator*{\argmin}{argmin}
\DeclareMathOperator*{\argmax}{argmax}
\DeclareMathOperator{\supp}{supp}
\DeclareMathOperator{\relint}{relint}
\DeclareMathOperator{\diag}{diag}
\DeclareMathOperator{\Diag}{Diag}
\DeclareMathOperator{\sgn}{sgn}
\DeclareMathOperator{\tr}{tr}
\renewcommand{\Re}{\text{Re}}

% macros
\newcommand{\suchthat}{\,:\,}
\newcommand{\abs}[1]{\lvert{#1}\rvert}
\newcommand{\card}[1]{\lvert{#1}\rvert}
\newcommand{\ceil}[1]{\lceil{#1}\rceil}
\newcommand{\floor}[1]{\lfloor{#1}\rfloor}
\newcommand{\define}{\coloneqq}
\newcommand{\enifed}{\eqqcolon}
\newcommand{\norm}[1]{\lVert{#1}\rVert}
\newcommand{\Norm}[2]{\lVert{#1}\rVert_{#2}}
\newcommand{\skal}[2]{\langle{#1},{#2}\rangle}
\newcommand{\order}[1]{O\left({#1}\right)}
\newcommand{\T}{^{\top}}

% sets
\newcommand{\R}{\mathds{R}}
\newcommand{\Z}{\mathds{Z}}
\newcommand{\Q}{\mathds{Q}}
\newcommand{\N}{\mathds{N}}
\newcommand{\C}{\mathds{C}}
\newcommand{\E}{\mathds{E}}
\newcommand{\ones}{\mathds{1}}
\renewcommand{\P}{\mathds{P}}
\renewcommand{\E}{\mathds{E}}

% environments
\newtheorem{theorem}{Theorem}
\newtheorem{lemma}[theorem]{Lemma}
\newtheorem{corollary}[theorem]{Corollary}
\newtheorem{definition}[theorem]{Definition}
\newtheorem{proposition}[theorem]{Proposition}
\newtheorem{observation}[theorem]{Observation}
\newtheorem{example}[theorem]{Example}
\newtheorem{question}[theorem]{Question}
\newtheorem{remark}[theorem]{Remark}
\newtheorem{assumption}[theorem]{Assumption}

% adjust itemize
\setlist[itemize]{topsep=0.5ex,partopsep=0ex,parsep=0ex,itemsep=0.5ex}

% other defines
\newcommand{\MP}[1]{\todo{#1}}
\newcommand{\MPin}[1]{\todo[inline]{#1}}


% title
\title{Termination Criteria for SDPs\\ in SCIP-SDP and Mosek}
\author{Marc E. Pfetsch\thanks{Department of Mathematics, TU Darmstadt, Germany}}
\date{April 2022}

% -------------------------------------------------------------------------
% -------------------------------------------------------------------------
\begin{document}

\maketitle

\noindent
Many parts of the following are based on the Mosek documentation.

% -------------------------------------------------------------------------
\section{Introduction}

Our goal is to solve the following \emph{dual} SDP problem:
\begin{equation}\label{SDP-D}
  \begin{aligned}
    \inf \quad & b\T y \\
    \text{s.t.} \quad & \sum_{j=1}^m A^j\, y_j - A^0 \succeq 0, \\
  \end{aligned}
\end{equation}
with symmetric matrices $A^j \in \R^{n \times n}$ for
$j \in [m]_0 \define \{0, \dots, m\}$ and $b \in \R^m$.  We define
$A(y) \define \sum_{j=1}^m A^j\, y_j - A^0$.

The corresponding dual problem, which we call \emph{primal} form, is the
following:
\begin{equation}\label{SDP-P}
  \begin{aligned}
     \sup \quad & \skal{A^0}{X} \\
     \mbox{s.t.} \quad & \skal{A^j}{X} = b_j && \forall \, j \in [m],\\
     & X \succeq 0.
  \end{aligned}
\end{equation}
This problem is solved when using Mosek.

% -------------------------------------------------------------------------
\section{Infeasibility in SDPs}

% We need the following result on infeasibility:
% \begin{lemma}\label{lemma:Farkas}
%   Either $A(y) + \varepsilon I \succeq 0$ is feasible for every
%   $\varepsilon > 0$ or there exists $X \succeq 0$ with $\skal{A^j}{X} = 0$,
%   $j \in [m]$, and $\skal{A^0}{X} > 0$.
% \end{lemma}
%
% For a proof, see, e.g., Kellner, Pfetsch, and Theobald: ``Irreducible
% Infeasible Subsystems of Semidefinite Systems'', Journal of Optimization
% Theory and Applications.
%
% We have the following two easy variants:

The following provides a proof of infeasibility of the dual problem~\eqref{SDP-D}:

\begin{lemma}\label{lemma:DualInfCertificate}
  If there exists $X \succeq 0$ with $\skal{A^0}{X} > 0$ and
  $\skal{A^j}{X} = 0$ for all $j \in [m]$, then there exists no $y$ with
  $A(y) \succeq 0$.
\end{lemma}

\begin{proof}
  Assume that $X \succeq 0$ as in the statement of the lemma exists and
  $A(y) \succeq 0$.  Then $\skal{A(y)}{X} \geq 0$. However,
  \[
  \skal{A(y)}{X} = \sum_{j=1}^m \underbrace{\skal{A^j}{X}}_{= 0}\, y_j -
  \underbrace{\skal{A^0}{X}}_{> 0} < 0,
  \]
  a contradiction.
\end{proof}

The following provides a proof of infeasibility of the primal problem~\eqref{SDP-P}:

\begin{lemma}\label{lemma:PrimalInfCertificate}
  If there exists $y$ with $\sum_{j=1}^m A^j y_j = 0$ and $b\T y \neq 0$, then
  there exists no $X \succeq 0$ with $\skal{A^j}{X} = b_j$ for all
  $j \in [m]$.
\end{lemma}

\begin{proof}
  Assume that $\sum_{j=1}^m A^j y_j = 0$, $b\T y \neq 0$ and there exists
  $X \succeq 0$ with $\skal{A^j}{X} = b_j$ for all $j \in [m]$. Then
  \[
  0 = \skal{\sum_{j=1}^m A^j y_j}{X} = \sum_{j=1}^m \skal{A^j}{X}\, y_j =
  b\T y \neq 0,
  \]
  a contradiction.
\end{proof}

We need the following variants:

\begin{lemma}\label{lemma:DualInfCertificate2}
  If there exists $X \succeq 0$ with $\skal{A^0}{X} > 0$ and
  $\skal{A^j}{X} = 0$ for all $j \in [m]$, then there exists no $(S,y)$
  with $A(y) - S = 0$, $S \succeq 0$.
\end{lemma}

\begin{proof}
  Assume that $X \succeq 0$ as in the statement of the lemma exists and
  $A(y) - S = 0$, $S \succeq 0$.  Then $\skal{A(y)}{X} \geq 0$. However,
  \[
  \skal{A(y)}{X} = \sum_{j=1}^m \underbrace{\skal{A^j}{X}}_{= 0}\, y_j -
  \underbrace{\skal{A^0}{X}}_{> 0} < 0,
  \]
  a contradiction.
\end{proof}

\begin{lemma}\label{lemma:PrimalInfCertificate2}
  If there exists $y$ with $\sum_{j=1}^m A^j y_j - S = 0$, $S \succeq 0$,
  and $b\T y < 0$, then there exists no $X \succeq 0$ with
  $\skal{A^j}{X} = b_j$ for all $j \in [m]$.
\end{lemma}

\begin{proof}
  Assume that $\sum_{j=1}^m A^j y_j - S = 0$, $S \succeq 0$, $b\T y < 0$,
  and there exists $X \succeq 0$ with $\skal{A^j}{X} = b_j$ for all
  $j \in [m]$. Then $\skal{S}{X} \geq 0$ and
  \[
  0 = \skal{\sum_{j=1}^m A^j y_j - S}{X} = \sum_{j=1}^m \skal{A^j}{X}\, y_j
  - \skal{S}{X} \leq b\T y < 0,
  \]
  a contradiction.
\end{proof}

% -------------------------------------------------------------------------
\section{SCIP-SDP}

SCIP-SDP uses absolute tolerances for determining feasibility for the
problems~\eqref{SDP-D} and~\eqref{SDP-P}.

\paragraph{Feasibility}

Primal feasibility of a point $\hat{X}$ is measured by SCIP-SDP as
\[
\abs{\skal{A^j}{\hat{X}} - b_j} \leq \varepsilon_p
\]
for all $j \in [m]$. Similarly, dual feasibility of $\hat{y}$ holds if
\[
\lambda_{\min}\big(A(\hat{y})\big) = \lambda_{\min}\Big(\sum_{j=1}^m A^j\, \hat{y}_j -
A^0\Big) \geq - \varepsilon_d,
\]
i.e., $A(\hat{y})$ is psd.


\paragraph{Optimality}

The optimality criterion of a point $(\hat{X}, \hat{y})$ is
\[
\abs{b\T \hat{y} - \skal{A^0}{\hat{X}}} \leq \varepsilon_g.
\]

\paragraph{Dual Infeasibility}

Dual infeasibility is detected if there exists $\hat{X}$ with
\[
\abs{\skal{A^j}{\hat{X}}} \leq \varepsilon_i \quad\text{and}\quad
\skal{A^0}{\hat{X}} > \varepsilon_i.
\]
This means $\hat{X}$ approximately satisfies the conditions of
Lemma~\ref{lemma:DualInfCertificate}.

\paragraph{Primal Infeasibility}

Primal infeasibility is detected if there exists $\hat{y}$ with
\[
\Big\lVert \sum_{j=1}^m A^j\, \hat{y}_j \Big\rVert_{\infty} \leq \varepsilon_i
\quad\text{and}\quad b\T \hat{y} < -\varepsilon_i \text{ or } b\T \hat{y} > \varepsilon_i.
\]
In this case, $\hat{y}$ approximately satisfies the conditions of
Lemma~\ref{lemma:PrimalInfCertificate}.

 Note that in the implementation $\varepsilon_i = \varepsilon_p = \varepsilon_d$.

% -------------------------------------------------------------------------
\section{Mosek}


% -------------------------------------------------------------------------
\subsection{Homogeneous Problem}

Mosek rewrites~\eqref{SDP-D} as
\begin{equation}\label{SDP-D2}
  \begin{aligned}
    \inf \quad & b\T y \\
    \text{s.t.} \quad & \sum_{j=1}^m A^j\, y_j - S = A^0, \\
    & S \succeq 0.
  \end{aligned}
\end{equation}
It then sets up a homogeneous model as
\begin{equation}\label{HOM}
  \begin{aligned}
    & \skal{A^j}{X} - b_j\,\tau = 0 && \forall \, j \in [m],\\
    & \sum_{j=1}^m A^j\, y_j - S - A^0\, \tau = 0, \\
    & \skal{A^0}{X} - b\T y - \kappa = 0,\\
    & X,\; S \succeq 0,\; \tau,\; \kappa \geq 0.
  \end{aligned}
\end{equation}
This problem is always feasible (all variables equal to 0 is feasible).

Any solution $(X,S,y,\kappa,\tau)$ satisfies
\[
\skal{X}{S} + \tau\, \kappa = 0,
\]
because
\begin{align*}
  \skal{X}{S} = \sum_{j=1}^m \skal{X}{A^j}\, y_j - \skal{X}{A^0} \tau = b\T y\,\tau - b\T y\,\tau - \kappa \tau.
\end{align*}
Thus, $\skal{X}{S} + \kappa\, \tau = 0$. Since $\skal{X}{S} \geq 0$,
$\kappa \, \tau = 0$ follows. Therefore either $\kappa = 0$ or $\tau = 0$.
A solution $(\hat{X},\hat{S},\hat{y},\hat{\kappa},\hat{\tau})$
of~\eqref{HOM} with $\hat{\tau} > 0$ (and thus $\hat{\kappa} = 0$) yields
an optimal solution
$(\frac{\hat{X}}{\hat{\tau}}, \frac{\hat{S}}{\hat{\tau}},
\frac{\hat{y}}{\hat{\tau}})$ of~\eqref{SDP-D} and~\eqref{SDP-P}:
\begin{align*}
  & \skal{A^j}{\tfrac{\hat{X}}{\hat{\tau}}} = b_j && \forall \, j \in [m],\\
  & \sum_{j=1}^m A^j\, \tfrac{\hat{y}_j}{\hat{\tau}} -
    A^0 = \tfrac{\hat{S}}{\hat{\tau}} \succeq 0, \\
  & \skal{A^0}{\tfrac{\hat{X}}{\hat{\tau}}} = b\T \tfrac{\hat{y}}{\hat{\tau}}.
\end{align*}

If $\hat{\kappa} > 0$ (and hence $\hat{\tau} = 0$), we have
$\skal{A^0}{\hat{X}} - b\T \hat{y} > 0$ and therefore
$\skal{A^0}{\hat{X}} > 0$ or $b\T \hat{y} < 0$. In the first case,
$\hat{X}$ satisfies the conditions of Lemma~\ref{lemma:DualInfCertificate2}
and is a certificate of dual infeasibility. In the second case,
$(\hat{S},\hat{y})$ satisfy the conditions of
Lemma~\ref{lemma:PrimalInfCertificate2} and form a certificate of primal
infeasibility.


% -------------------------------------------------------------------------
\subsection{Termination Criteria}

\paragraph{Feasibility}

Primal feasibility of a point
$(\hat{X},\hat{S},\hat{y},\hat{\kappa},\hat{\tau})$ is measured by Mosek as
\[
\abs{\skal{A^j}{\tfrac{\hat{X}}{\hat{\tau}}} - b_j}
\leq \varepsilon_p (1 + \norm{b}_\infty)
\]
for all $j \in [m]$. Similarly, dual feasibility holds if
\[
\Big\lVert\sum_{j=1}^m A^j\, \tfrac{\hat{y}_j}{\hat{\tau}} - \tfrac{\hat{S}}{\hat{\tau}} - A^0\Big\rVert_\infty
\leq \varepsilon_d (1 + \norm{A^0}_\infty).
\]
In addition the fact that $\hat{X}$ and $\hat{S}$ are psd is checked by an
undocumented method.

\paragraph{Optimality}

The optimality criterion is
\[
\min\Big( \skal{\tfrac{\hat{X}}{\hat{\tau}}}{\tfrac{\hat{S}}{\hat{\tau}}},
\abs{b\T \tfrac{\hat{y}}{\hat{\tau}} - \skal{A^0}{\tfrac{\hat{X}}{\hat{\tau}}}}\Big)
\leq \varepsilon_g \max\Big(1,
\min(\abs{\skal{A^0}{\tfrac{\hat{X}}{\hat{\tau}}}}, \abs{b\T
  \tfrac{\hat{y}}{\hat{\tau}}})\Big).
\]
In fact, the Mosek API suggests that there are two tolerances for each term
on the left. So probably, a solution is claimed to be optimal if
\[
\skal{\tfrac{\hat{X}}{\hat{\tau}}}{\tfrac{\hat{S}}{\hat{\tau}}} \leq
\varepsilon_g' G
\quad\text{or}\quad
\abs{b\T \tfrac{\hat{y}}{\hat{\tau}} - \skal{A^0}{\tfrac{\hat{X}}{\hat{\tau}}}}
\leq \varepsilon_g'' G,
\]
with
\[
G \define \max\big(1, \min(\abs{\skal{A^0}{\tfrac{\hat{X}}{\hat{\tau}}}},
\abs{b\T \tfrac{\hat{y}}{\hat{\tau}}})\big).
\]

\paragraph{Primal Infeasibility}

Primal infeasibility is detected if
\[
\Big\lVert \sum_{j=1}^m A^j\, \hat{y}_j + \hat{S} \Big\rVert_{\infty} \leq -\varepsilon_i\, b\T \hat{y}
\quad\text{and}\quad b\T \hat{y} < 0.
\]
In this case, we have for $\bar{y} = \frac{\hat{y}}{\alpha}$, $\bar{S}
= \frac{\hat{S}}{\alpha}$ with $\alpha \define -b\T \hat{y}$:
\[
\Big\lVert \sum_{j=1}^m A^j\, \bar{y}_j + \bar{S}\Big\rVert_{\infty} \leq \varepsilon_i
\quad\text{and}\quad b\T \bar{y} = -1.
\]
This means that $(\bar{y},\bar{S})$ is (approximately) feasible for the
recession cone $\{(y,S) \suchthat \sum_{j=1}^m A^j\, y + S = 0,\; S \succeq 0\}$ (note that
$A^0 = 0$ in this case, so the above dual feasibility check applies).

Then $(\hat{y} + \lambda\, \bar{y}, \hat{S} + \lambda\, \bar{S})$ is
``approximately'' dual feasible for any $\lambda \geq 0$, because
\begin{align*}
  & \Big\lVert \sum_{j=1}^m A^j\, (\hat{y}_j + \lambda\, \bar{y}_j) + \hat{S} + \lambda\,
    \bar{S} - A^0 \Big\rVert_{\infty}\\
  =\; & \Big\lVert \sum_{j=1}^m A^j\, \hat{y}_j + \hat{S} - A^0 + \lambda\, \Big(\sum_{j=1}^m
    A^j \bar{y}_j + \bar{S}\Big)\Big\rVert_{\infty}\\
  \leq\; & \varepsilon_d (1 + \norm{A^0}_\infty) + \lambda\, \varepsilon_i.
\end{align*}
Moreover
$b\T (\hat{y} + \lambda\, \bar{y}) = b\T \hat{y} - \lambda \to -\infty$ for
$\lambda \to \infty$.

\paragraph{Dual Infeasibility}

Similarly, dual infeasibility is detected if
\[
\abs{\skal{A^j}{\hat{X}}} \leq \varepsilon_i\, \skal{A^0}{\hat{X}}
\quad\text{and}\quad \skal{A^0}{\hat{X}} > 0.
\]
In this case, we have for $\bar{X} = \frac{\hat{X}}{\gamma}$ with
$\gamma \define \skal{A^0}{\hat{X}}$:
\[
\abs{\skal{A^j}{\bar{X}}} \leq \varepsilon_i
\quad\text{and}\quad \skal{A^0}{\hat{X}} = 1.
\]
This means that $\bar{X}$ is (approximately) feasible for the recession
cone $\{X \suchthat \skal{A^j}{X} = 0\; \forall j \in [m],\; X \succeq 0\}$
(note that $A^0 = 0$ in this case, so the above primal feasibility check
applies).

Then $\hat{X} + \lambda\, \bar{X}$ is ``approximately'' primal feasible
for any $\lambda \geq 0$, because
\begin{align*}
  \abs{\skal{A^j}{\hat{X} + \lambda\, \bar{X}} - b_j} =
  \abs{\skal{A^j}{\hat{X}} - b_j + \lambda\, \skal{A^j}{\bar{X}}}
  \leq \varepsilon_p (1 + \norm{b}_{\infty}) + \lambda\, \varepsilon_i.
\end{align*}
Moreover
$\skal{A^0}{\hat{X} + \lambda\, \bar{X}} = \skal{A^0}{\hat{X}} + \lambda \to \infty$ for
$\lambda \to \infty$.



% -------------------------------------------------------------------------
\subsection{Scaling}

Assume that $b$ in~\eqref{SDP-D} (or \eqref{SDP-D2}) is replaced by
$\tilde{b} = \beta\, b$ for some $\beta > 0$. Then the set of optimal
solutions to these problems is the same. In fact, if
$(\hat{X},\hat{y},\hat{S})$ is a primal/dual optimal point, then
$(\hat{y},\hat{S})$ remains feasible for~\eqref{SDP-D2} and $\beta\, \hat{X}$ is
feasible for~\eqref{SDP-P} with $\tilde{b}$ instead of $b$:
\[
\skal{A^j}{\beta\, \hat{X}} = \beta\, \skal{A^j}{\hat{X}} = \beta\,b_j = \tilde{b}_j.
\]
If $\beta < 1$, this solution is also numerically feasible:
\[
\abs{\skal{A^j}{\beta\, \hat{X}} - \tilde{b}_j}
= \beta \abs{\skal{A^j}{\hat{X}} - b_j}
\leq \beta \varepsilon_p (1 + \norm{b}_\infty)
= \varepsilon_p (\beta + \norm{\tilde{b}}_\infty) \leq \varepsilon_p (1 + \norm{\tilde{b}}_\infty).
\]
Conversely, assume that $\tilde{X}$ is feasible for~\eqref{SDP-P} with
$\tilde{b}$ instead of $b$. Then $\hat{X} = \tilde{X}/\beta$ is numerically
feasible for the original~\eqref{SDP-P} if $\beta > 1$.
\[
\abs{\skal{A^j}{\hat{X}} - b_j}
= \tfrac{1}{\beta} \abs{\skal{A^j}{\tilde{X}} - \tilde{b}_j}
\leq \tfrac{1}{\beta} \varepsilon_p (1 + \norm{\tilde{b}}_\infty)
= \varepsilon_p (\tfrac{1}{\beta} + \norm{b}_\infty) \leq \varepsilon_p (1 + \norm{b}_\infty).
\]

Unfortunately, the above considerations show that one cannot simply adjust
$\varepsilon_i$ to have an equivalence between the original and the scaled
problem (note that the norms $\norm{b}_{\infty}$ and
$\norm{\tilde{b}}_{\infty}$ are computed by Mosek and we cannot control
them).


\end{document}


% -------------------------------------------------------------------------
\section{Introduction}

Our goal is to solve the following dual SDP problem:
\begin{equation}\label{SDP-D}
  \begin{aligned}
    \inf \quad & b\T y \\
    \text{s.t.} \quad & \sum_{j=1}^m A^j\, y_j - A^0 \succeq 0, \\
    & D\, y \geq d,\\
    & \ell \leq y \leq u.
  \end{aligned}
\end{equation}
with symmetric matrices $A^j \in \R^{n \times n}$ for
$j \in [m]_0 \define \{0, \dots, m\}$, $b$, $\ell$, $u \in \R^m$. Moreover,
$D \in \R^{r \times m}$, $d \in \R^r$. Note that we assume that the bounds
are finite for simplicity. The formulas below can be easily modified if
they are infinite.

In Mosek, we solve the corresponding dual problem, which we call ``primal''
form in the following:
\begin{equation}\label{SDP-P}
  \begin{aligned}
     \sup \quad & \skal{A^0}{X} + d\T x - u\T v + \ell\T w \\
     \mbox{s.t.} \quad & \skal{A^j}{X} + D_j\T x - v_j + w_j = b_j && \forall \, j \in [m],\\
     & x, v, w \geq 0,\\
     & X \succeq 0.\\
  \end{aligned}
\end{equation}

% -------------------------------------------------------------------------
\section{Homogeneous Problem}

Mosek rewrites~\eqref{SDP-D} as
\begin{equation}\label{SDP-D2}
  \begin{aligned}
    \inf \quad & b\T y \\
    \text{s.t.} \quad & \sum_{j=1}^m A^j\, y_j - S = A^0, \\
    & D\, y - d - s = 0,\\
    & y - \ell - v = 0,\\
    & u - y - w = 0,\\
    & S \succeq 0,\; s, v, w \geq 0.
  \end{aligned}
\end{equation}
Then the homogeneous model is
\begin{equation}\label{HOM}
  \begin{aligned}
    & \skal{A^j}{X} + D_j\T x - v_j + w_j - b_j\,\tau = 0 && \forall \, j \in [m],\\
    & \sum_{j=1}^m A^j\, y_j - S - A^0 \tau = 0, \\
    & D\, y - s - d \tau = 0,\\
    & y - v - \ell \tau = 0,\\
    & y + w - u \tau = 0,\\
    & \skal{A^0}{X} + d\T x - u\T v + \ell\T w - b\T y - \kappa = 0,\\
    & X, S \succeq 0,\\
    & s, v, w, \tau, \kappa \geq 0.
  \end{aligned}
\end{equation}
Any solution $(X,S,s,x,y,v,w,\kappa,\tau)$ satisfies
\[
\skal{X}{S} + \tau\, \kappa = 0,
\]
because
\begin{align*}
  \skal{X}{S} & = \sum_{j=1}^m \skal{X}{A^j}\, y_j - \skal{X}{A^0} \tau\\
  & = \sum_{j=1}^m (b_j\,\tau - w_j + v_j - D_j\T x)\, y_j + (d\T x - u\T v + \ell\T w - b\T y - \kappa) \tau\\
  & = b\T y\,\tau - w\T y + v\T y - x\T D y + d\T x\,\tau - u\T v\,\tau +
    \ell\T w\, \tau - b\T y\,\tau - \kappa \tau\\
  & = - w\T y + v\T y - x\T D y + (D y - s)\T x - (y + w)\T v +
    (y - v)\T w - \kappa \tau\\
  & = - w\T y + v\T y - x\T D y + x\T D y - s\T x - y\T v  + w\T v +
    y\T w - v\T w - \kappa \tau\\
  & =   -s\T x - \kappa \tau.
\end{align*}
Thus, $\skal{X}{S} + s\T x + \kappa\, \tau = 0$. Since $\skal{X}{S} + s\T x
\geq 0$, $\kappa \, \tau = 0$ follows.


%%% Local Variables:
%%% mode: latex
%%% TeX-master: t
%%% End:
