\documentclass[10pt, a4paper]{article}
\usepackage[utf8]{inputenc}
\usepackage[T1]{fontenc}
\usepackage[american]{babel}
\usepackage[dvipsnames]{xcolor}
\usepackage{amsmath}
\usepackage{amssymb}
\usepackage{amsthm}
\usepackage{dsfont}
\usepackage{mathtools}
\usepackage{graphicx}
\usepackage{enumitem}
\usepackage{booktabs}
\usepackage{siunitx}
\usepackage[lined,ruled,noend,linesnumbered]{algorithm2e}
\usepackage[color=red!50!blue!50,textsize=footnotesize,textwidth=4cm]{todonotes}
\usepackage[breaklinks,colorlinks,citecolor=blue,linkcolor=blue]{hyperref}
% \usepackage[tmargin=2.5cm,bmargin=2.5cm,lmargin=2.8cm,rmargin=2.8cm]{geometry}
\usepackage{cleveref}
\usepackage{xspace}

% operators
\DeclareMathOperator{\conv}{conv}
\DeclareMathOperator{\cone}{cone}
\DeclareMathOperator{\aff}{aff}
\DeclareMathOperator{\lin}{lin}
\DeclareMathOperator*{\argmin}{argmin}
\DeclareMathOperator*{\argmax}{argmax}
\DeclareMathOperator{\supp}{supp}
\DeclareMathOperator{\relint}{relint}
\DeclareMathOperator{\diag}{diag}
\DeclareMathOperator{\Diag}{Diag}
\DeclareMathOperator{\sgn}{sgn}
\DeclareMathOperator{\tr}{tr}
\renewcommand{\Re}{\text{Re}}

% macros
\newcommand{\suchthat}{\,:\,}
\newcommand{\abs}[1]{\lvert{#1}\rvert}
\newcommand{\card}[1]{\lvert{#1}\rvert}
\newcommand{\ceil}[1]{\lceil{#1}\rceil}
\newcommand{\floor}[1]{\lfloor{#1}\rfloor}
\newcommand{\define}{\coloneqq}
\newcommand{\enifed}{\eqqcolon}
\newcommand{\norm}[1]{\lVert{#1}\rVert}
\newcommand{\Norm}[2]{\lVert{#1}\rVert_{#2}}
\newcommand{\skal}[2]{\langle{#1},{#2}\rangle}
\newcommand{\order}[1]{O\left({#1}\right)}
\newcommand{\T}{^{\top}}

% sets
\newcommand{\R}{\mathds{R}}
\newcommand{\Z}{\mathds{Z}}
\newcommand{\Q}{\mathds{Q}}
\newcommand{\N}{\mathds{N}}
\newcommand{\C}{\mathds{C}}
\newcommand{\E}{\mathds{E}}
\newcommand{\ones}{\mathds{1}}
\renewcommand{\P}{\mathds{P}}
\renewcommand{\E}{\mathds{E}}

% environments
\newtheorem{theorem}{Theorem}
\newtheorem{lemma}[theorem]{Lemma}
\newtheorem{corollary}[theorem]{Corollary}
\newtheorem{definition}[theorem]{Definition}
\newtheorem{proposition}[theorem]{Proposition}
\newtheorem{observation}[theorem]{Observation}
\newtheorem{example}[theorem]{Example}
\newtheorem{question}[theorem]{Question}
\newtheorem{remark}[theorem]{Remark}
\newtheorem{assumption}[theorem]{Assumption}

% adjust itemize
\setlist[itemize]{topsep=0.5ex,partopsep=0ex,parsep=0ex,itemsep=0.5ex}

% other defines
\newcommand{\MP}[1]{\todo{#1}}
\newcommand{\MPin}[1]{\todo[inline]{#1}}


% title
\title{Conflict Analysis for Mixed-integer SDPs}
\author{Marc E. Pfetsch\thanks{Department of Mathematics, TU Darmstadt, Germany}}
\date{June 2022}

% -------------------------------------------------------------------------
% -------------------------------------------------------------------------
\begin{document}

\maketitle



% -------------------------------------------------------------------------
\section{Introduction}

We consider mixed-integer SDPs of the following form:
\begin{equation}\label{MISDP}
  \begin{aligned}
    \inf \quad & b\T y \\
    \text{s.t.} \quad & \sum_{j=1}^m A^j\, y_j - A^0 \succeq 0, \\
    & D y \geq d,\\
    & \ell \leq y \leq u,\\
    & y_i \in \Z,\quad i \in I.
  \end{aligned}
\end{equation}
In each node, we solve the following relaxations with local bounds $\ell$
and $u$:
\begin{equation}\label{SDP-D}
  \begin{aligned}
    \inf \quad & b\T y \\
    \text{s.t.} \quad & \sum_{j=1}^m A^j\, y_j - A^0 \succeq 0, \\
    & D y \geq d,\\
    & \ell \leq y \leq u,
  \end{aligned}
\end{equation}
where for simplicity of exposition, we assume that all entries of $\ell$
and $u$ are finite. Moreover, we always assume that $\ell \leq u$ holds.

Let $X$ be the dual variables for the SDP-constraint $A(y) \succeq 0$, $z$
be the dual variables for $Dy \geq d$, and $r^\ell$ and $r^u$ be the dual
variables for the bound constraints $y \geq \ell$ and $-y \geq -u$,
respectively. Then the corresponding primal problem for~\eqref{SDP-D} is:
\begin{equation}\label{SDP-P}
  \begin{aligned}
     \sup \quad & \skal{A^0}{X} + z\T d + \ell\T r^\ell - u\T r^u\\
     \text{s.t.} \quad & \skal{A^j}{X} + (D\T z)_j + r^\ell_j - r^u_j = b_j && \forall \, j \in [m],\\
     & X \succeq 0,\; z, \; r^\ell,\; r^u \geq 0.
  \end{aligned}
\end{equation}

We will need the following result.

\begin{lemma}\label{lemma:ComplemenatrySolution}
  Let $\hat{r}^\ell$, $\hat{r}^u \in \R^m_+$ with
  $\hat{r}^\ell - \hat{r}^u = \hat{s}$ for some $\hat{s} \in \R^m$. Then
  there exists $r^\ell$, $r^u \in \R^m_+$ with $r^\ell - r^u = \hat{s}$, such
  that $r_j^\ell \cdot r_j^u = 0$ for all $j \in [m]$ and for
  $\ell \leq u$:
  \begin{equation}\label{eq:objpart}
    \ell\T r^\ell - u\T r^u \geq \ell\T \hat{r}^\ell - u\T \hat{r}^u.
  \end{equation}
\end{lemma}

\begin{proof}
  Define
  \[
    r^\ell_j \define - \min\,\{\hat{s}_j, 0\} \geq 0,
    \quad
    r^u_j = \max\,\{\hat{s}_j, 0\} \geq 0.
  \]
  By construction $r^\ell_j \cdot r^u_j = 0$ for all
  $j \in [m]$ and $r^\ell - r^u = \hat{s}$. Then consider:
  \begin{align*}
    \ell_j r^\ell_j - u_j r^u_j & = - \ell_j\, \min\,\{\hat{s}_j, 0\} - u_j\, \max\,\{\hat{s}_j, 0\}\\
    & = - \ell_j\, \min\,\{\hat{r}^u_j - \hat{r}^\ell_j, 0\} - u_j\, \max\,\{\hat{r}^u_j - \hat{r}^\ell_j, 0\}\\
    & = \begin{cases}
      - \ell_j\, (\hat{r}^u_j - \hat{r}^\ell_j) & \text{if } \hat{r}^u_j - \hat{r}^\ell_j < 0,\\
      - u_j\, (\hat{r}^u_j - \hat{r}^\ell_j) & \text{if } \hat{r}^u_j - \hat{r}^\ell_j \geq 0,
    \end{cases}\\
    & = \begin{cases}
      - \ell_j\, \hat{r}^u_j + \ell_j \hat{r}^\ell_j & \text{if } \hat{r}^u_j - \hat{r}^\ell_j < 0,\\
      - u_j\, \hat{r}^u_j + u_j \hat{r}^\ell_j & \text{if } \hat{r}^u_j - \hat{r}^\ell_j \geq 0,
    \end{cases}\\
    & \geq \begin{cases}
      - u_j\, \hat{r}^u_j + \ell_j \hat{r}^\ell_j & \text{if }  \hat{r}^u_j - \hat{r}^\ell_j < 0,\\
      - u_j\, \hat{r}^u_j + \ell_j \hat{r}^\ell_j & \text{if } \hat{r}^u_j - \hat{r}^\ell_j \geq 0,
    \end{cases}\\
    & = \ell_j \hat{r}^\ell_j - u_j \hat{r}^u_j,
  \end{align*}
  which shows~\eqref{eq:objpart}.
\end{proof}

% -------------------------------------------------------------------------
% -------------------------------------------------------------------------
\section{Brief Literature Review for Conflict Analysis}

The original intention of conflict analysis was to derive useful
information from infeasible subproblems in a branch-and-bound
procedure. There are three types of conflict analysis:
\begin{itemize}
\item If infeasibility was determined using propagation of variable bounds,
  one can use a \emph{graph-based conflict analysis}. Here, a directed graph
  encodes the dependency of bound changes on each other, i.e., whether one
  bound was used to strengthen the other bound. One can then try to find a
  small subset of bound changes that arise from branching decisions that
  still lead to infeasibility. One can then add constraints enforcing
  that at least one branching decision is taken in a different
  way. See~\cite{Ach07b,Ach07} for more information in the context of MIPs.

  This kind of conflict analysis can also be used in the context of MISDPs
  and has already been investigated in~\cite{MatP22}. Overall, this kind of
  conflict analysis was not beneficial.
\item If the relaxation is infeasible, there is a \emph{greedy approach},
  which relaxes some of the branching decisions defining the current node
  as long as the relaxation remains infeasible. Then a constraint as in
  graph-based conflict analysis can be derived. See~\cite{Ach07b,Ach07} for
  more information in the context of MIPs. This method does not seem very
  promising for MISDPs, since we do not have warm start in this case.
\item If the relaxation is infeasible, one can also use dual information
  to derive valid inequalities, which can then be propagated during the
  tree. This is called dual ray analysis by Witzig~\cite{Wit22} (see also
  \cite{WitBH17}).
\end{itemize}

In the following, we will generalize the last approach to MISDPs and call
it \emph{dual conflict analysis}.

Very similar ideas are needed for an outer approximation algorithm, in
which one adds aggregated inequalities that will cut off nodes that are
infeasible or bound exceeding. Coey et al.~\cite{CoeLV20} describe such an
algorithm for mixed-integer conic problems (including MISDP).


% -------------------------------------------------------------------------
% -------------------------------------------------------------------------
\section{Generation of Conflict Constraints}

The general idea of dual conflict analysis is to generate globally valid
constraints as follows. Let any $\hat{X} \succeq 0$ and $\hat{z} \geq 0$ be
given. Then because $A(y) \define \sum_{j=1}^m A^j\, y_j - A^0$ is positive
semidefinite for all feasible $y \in \R^m$, we have
$\skal{A(y)}{\hat{X}} \geq 0$.  Similarly, the aggregated inequalities
$\hat{z}\T D y \geq \hat{z}\T d$ hold. Together this yields:
\begin{align}
  & \skal{A(y)}{\hat{X}} + \hat{z}\T D y \geq \hat{z}\T d\notag\\
  \Leftrightarrow\quad
  & \sum_{j=1}^m \skal{A^j}{\hat{X}} y_j + \hat{z}\T D y \geq
    \skal{A^0}{\hat{X}} + \hat{z}\T d,\label{eq:ConflictCut}
\end{align}
which is valid for all feasible $y$. Clearly,
Inequality~\eqref{eq:ConflictCut} is redundant for continuous $y$, but it
might allow to cut off certain mixed-integer solutions~$y$. In the implementation,
we only propagate bounds using~\eqref{eq:ConflictCut}.

More generally, we can also use the variable bounds $\ell \leq y$ and
$y \leq u$. If $\hat{r}^\ell$, $\hat{r}^u \geq 0$ are the corresponding
multipliers, this yields:
\begin{align}
  & \skal{A(y)}{\hat{X}} + \hat{z}\T D y + (\hat{r}^\ell)\T y - (\hat{r}^u)\T y
    \geq \hat{z}\T d + (\hat{r}^\ell)\T \ell - (\hat{r}^u)\T u\notag\\
  \Leftrightarrow\quad
  & \sum_{j=1}^m \skal{A^j}{\hat{X}} y_j + \hat{z}\T D y +
    (\hat{r}^\ell)\T y - (\hat{r}^u)\T y \geq \\
  & \qquad \skal{A^0}{\hat{X}} +
    \hat{z}\T d + (\hat{r}^\ell)\T \ell - (\hat{r}^u)\T u,\label{eq:ConflictCut2}
\end{align}
which is again valid for all feasible $y$.

It seems that Inequalities~\eqref{eq:ConflictCut2} are less helpful
than~\eqref{eq:ConflictCut}. However, one can cancel some coefficients
in~\eqref{eq:ConflictCut} using variable bounds -- see
Section~\ref{sec:Canceling}.

% -------------------------------------------------------------------------
\subsection{Infeasible Problems}

A primal ray, i.e., a point $(\hat{X}, \hat{z}, \hat{r}^\ell, \hat{r}^u)$
satisfying the system
\begin{equation}\label{eq:ray}
\begin{aligned}
  & \skal{A^j}{X} + (D\T z)_j + r^\ell_j - r^u_j = 0 && \forall\; j \in [m],\\
  & \skal{A^0}{X} + d\T z + \ell\T r^\ell - u\T r^u > 0,
\end{aligned}
\end{equation}
\todo[inline]{Frederic: I think we also need~$X \succeq 0$ and~$z,\, r^{\ell},\, u
  \geq 0$ in~\eqref{eq:ray}.}
yields an inequality as in~\eqref{eq:ConflictCut}
or~\eqref{eq:ConflictCut2}.

In this case, one can prove that~\eqref{eq:ConflictCut} actually proves
infeasibility for the local bounds~$\ell$ and $u$ via propagation:

\begin{proposition}
  Let $(\hat{X}, \hat{z}, \hat{r}^\ell, \hat{r}^u)$ be a solution
  of~\eqref{eq:ray}. Then~\eqref{eq:ConflictCut}, i.e.,
  \[
    \skal{A(y)}{\hat{X}} + \hat{z}\T D \geq \hat{z}\T d
  \]
  is infeasible with respect to the local bounds $\ell$ and $u$.
\end{proposition}

\begin{proof}
  One the one hand, any feasible solution $\hat{y}$ of~\eqref{MISDP} satisfies
  $A(\hat{y}) \succeq 0$, $D \hat{y} \geq d$, and
  $\ell \leq \hat{y} \leq u$. Defining
  $\hat{s}_j \define \skal{A^j}{\hat{X}} + (\hat{z}\T D)_j$ for
  $j \in [m]$, we get from~\eqref{eq:ConflictCut}:
  \begin{align*}
    \skal{A^0}{\hat{X}} + \hat{z}\T d & \leq \sum_{j=1}^m \big(\skal{A^j}{\hat{X}} + (\hat{z}\T D)_j\big)\, \hat{y}_j
    = \sum_{j=1}^m \hat{s}_j\, \hat{y}_j\\
    & \leq \sum_{j: \hat{s}_j < 0} \hat{s}_j\, \ell_j +
        \sum_{j: \hat{s}_j > 0} \hat{s}_j\, u_j.
  \end{align*}

  On the other hand, by applying Lemma~\ref{lemma:ComplemenatrySolution} to
  $\hat{r}^\ell - \hat{r}^u = -\hat{s}$, we can assume that at most one of
  $\hat{r}^\ell_j$ and $\hat{r}^u_j$ is nonzero and the inequality
  in~\eqref{eq:ray} is still satisfied. We thus obtain
  \begin{align*}
    0 & < \skal{A^0}{\hat{X}} + d\T \hat{z} + \ell\T \hat{r}^\ell - u\T \hat{r}^u\\
      & = \skal{A^0}{\hat{X}} + d\T \hat{z} + \sum_{j=1}^m \ell_j \hat{r}^\ell_j -
        \sum_{j=1}^m u_j \hat{r}^u_j\\
      & = \skal{A^0}{\hat{X}}  + d\T \hat{z} - \sum_{j: \hat{s}_j < 0} \ell_j\, \hat{s}_j -
        \sum_{j: \hat{s}_j > 0} u_j\, \hat{s}_j,
  \end{align*}
  a contradiction.
\end{proof}

The hope is that~\eqref{eq:ConflictCut} is still violated for local bounds
not far from $\ell \leq u$.

% -------------------------------------------------------------------------
\subsection{Feasible Problems}

Any feasible solution $(\hat{X}, \hat{z}, \hat{r}^\ell, \hat{r}^u)$
of~\eqref{SDP-P} can be used to construct~\eqref{eq:ConflictCut}
or~\eqref{eq:ConflictCut2}.

Moreover, one can take an objective cutoff into account. Namely, assume
that $y^\star$ is a feasible solution to~\eqref{MISDP}. We are then only
interested in solutions that are at least as good, i.e.,
$b\T y \leq \overline{b}$ or equivalently $- b\T y \geq - \overline{b}$,
where $\overline{b} \define b\T y^\star - \varepsilon$ for
$\varepsilon > 0$. Adding this inequality to~\eqref{eq:ConflictCut} yields:
\begin{align}
  & \sum_{j=1}^m \skal{A^j}{\hat{X}} y_j + \hat{z}\T D y - b\T y \geq
    \skal{A^0}{\hat{X}} + \hat{z}\T d - \overline{b}\notag\\
  \Leftrightarrow\quad
  & \sum_{j=1}^m \big(\skal{A^j}{\hat{X}} + (\hat{z}\T D)_j - b_j\big) y_j \geq
    \skal{A^0}{\hat{X}} + \hat{z}\T d - \overline{b}.\label{eq:ConflictCutObj}
\end{align}

\begin{proposition}
  Let $(\hat{X}, \hat{z}, \hat{r}^\ell, \hat{r}^u)$ be a solution
  of~\eqref{SDP-P} and $\hat{y}$ be a mixed-integer local solution of the
  corresponding dual problem. Assume that strong duality holds. Define
  $\overline{b} \define b\T \hat{y} - \varepsilon$ for $\varepsilon >
  0$. Then~\eqref{eq:ConflictCutObj}, i.e.,
  \[
    \sum_{j=1}^m \big(\skal{A^j}{\hat{X}} + (\hat{z}\T D)_j - b_j\big) y_j \geq
    \skal{A^0}{\hat{X}} + \hat{z}\T d - \overline{b}.
  \]
  is infeasible with respect to the local bounds $\ell$ and $u$.
\end{proposition}

\begin{proof}
  Define $\hat{s}_j \define \skal{A^j}{\hat{X}} + (\hat{z}\T D)_j - b_j$
  for $j \in [m]$. Then we get from~\eqref{eq:ConflictCutObj} for any
  feasible solution $y$ with $b\T y \leq \overline{b}$ with $\ell \leq y
  \leq u$:
  \begin{align*}
    \skal{A^0}{\hat{X}} + \hat{z}\T d - \overline{b} & \leq \sum_{j=1}^m \big(\skal{A^j}{\hat{X}} + (\hat{z}\T D)_j - b_j\big)\, y_j
    = \sum_{j=1}^m \hat{s}_j\, y_j\\
    & \leq \sum_{j: \hat{s}_j < 0} \hat{s}_j\, \ell_j +
        \sum_{j: \hat{s}_j > 0} \hat{s}_j\, u_j.
  \end{align*}

  On the other hand, by applying Lemma~\ref{lemma:ComplemenatrySolution} to
  $\hat{r}^\ell - \hat{r}^u = -\hat{s}$, we can assume that at most one of
  $\hat{r}^\ell_j$ and $\hat{r}^u_j$ is nonzero and the corresponding
  solution is still optimal. We thus obtain by strong duality:
  \begin{align*}
    b\T \hat{y} & = \skal{A^0}{\hat{X}} + d\T \hat{z} + \ell\T \hat{r}^\ell - u\T \hat{r}^u\\
      & = \skal{A^0}{\hat{X}} + d\T \hat{z} + \sum_{j=1}^m \ell_j \hat{r}^\ell_j -
        \sum_{j=1}^m u_j \hat{r}^u_j\\
      & = \skal{A^0}{\hat{X}}  + d\T \hat{z} - \sum_{j: \hat{s}_j < 0} \ell_j\, \hat{s}_j -
        \sum_{j: \hat{s}_j > 0} u_j\, \hat{s}_j.
  \end{align*}
  This means that
  \begin{align*}
    \sum_{j: \hat{s}_j < 0} \ell_j\, \hat{s}_j +
        \sum_{j: \hat{s}_j > 0} u_j\, \hat{s}_j = \skal{A^0}{\hat{X}}  +
    d\T \hat{z} - \overline{b} - \varepsilon
  \end{align*}
  and therefore
  \begin{align*}
    \sum_{j: \hat{s}_j < 0} \ell_j\, \hat{s}_j +
        \sum_{j: \hat{s}_j > 0} u_j\, \hat{s}_j < \skal{A^0}{\hat{X}}  +
    d\T \hat{z} - \overline{b},
  \end{align*}
  a contradiction.
\end{proof}

The hope is that~\eqref{eq:ConflictCutObj} is still violated for local
bounds not far from $\ell \leq u$.

% -------------------------------------------------------------------------
\subsection{Canceling}
\label{sec:Canceling}

Starting from~\eqref{eq:ConflictCut}, one can possibly cancel coefficients
as follows. Again using
$\hat{s}_j \define \skal{A^j}{\hat{X}} + (\hat{z}\T D)_j$, we get
from~\eqref{eq:ConflictCut}:
\begin{equation}\label{eq:ConflictCut3}
  \sum_{j=1}^m \hat{s}_j\, y_j \geq \skal{A^0}{\hat{X}} + \hat{z}\T d.
\end{equation}

Assume that $\hat{s}_j < 0$ for some $j \in [m]$ and the global lower bound
of $y_j$ is at least 0. Then we can add the valid inequality
$-\hat{s}_j\, y_j \geq 0$ to~\eqref{eq:ConflictCut3}, which eliminates
variable~$y_j$ from the inequality. Similarly, if $\hat{s}_j > 0$ for some
$j \in [m]$ and the global upper bound of $y_j$ is at most 0, we can add
$\hat{s}_j\, y_j \leq 0$ to eliminate variable~$y_j$.

Witzig et al.~\cite{WitBH17} (see also Witzig~\cite{Wit22}) apply
cancellation to continuous variables, but restrict attention to variables
that are at their global bounds at the node at which the constraint is
generated.


% -------------------------------------------------------------------------
\subsection{Primal Solutions for Special Cases}
\label{sec:Primal}

In order to produce conflict constraints, one needs solutions for the
primal of the LP-relaxation~\eqref{SDP-P}. Because SCIP-SDP preprocesses
certain SDP problems, we briefly discuss how primal solutions can be
obtained in these cases.

% -------------------------------------------------------------------------
\subsubsection{SDPs with One Variable}

For the special case of SDP-relaxations in which only one variable is
unfixed, SCIP-SDP uses specialized methods to solve them. In order to
derive a primal solution, consider
\begin{equation}\label{eq:basic}
  \inf\; \{\gamma\, \mu \suchthat \mu\, A - B \succeq 0,\; \ell \leq \mu \leq u\},
\end{equation}
where $A$, $B \in \R^{n \times n}$ are symmetric matrices, $\gamma \in \R$,
and $\ell \in \R \cup \{-\infty\}$, $u \in \R \cup \{\infty\}$.

The dual SDP of~\eqref{eq:basic} is
\begin{equation}\label{eq:dual}
  \sup\; \{\skal{B}{X} + \ell\, \alpha - u\, \beta \suchthat \skal{A}{X} +
  \alpha - \beta = \gamma,\; X \succeq 0,\; \alpha,\; \beta \geq 0\}.
\end{equation}

If $\mu^*$ is an optimal solution for~\eqref{eq:basic}, we can construct an
optimal primal solution $(X^*, \alpha^*, \beta^*)$ as follows.

\paragraph{Case 0, $\boldsymbol{\gamma = 0}$:}

Here we can take $(X^*, \alpha^*, \beta^*) = (0,0,0)$. Thus, in the
following, we assume that $\gamma \neq 0$.

\paragraph{Case 1, $\boldsymbol{\ell < \mu^* < u}$:}

Then by complementary slackness necessarily $\alpha^* = \beta^* = 0$, and
thus $\skal{A}{X^*} = \gamma$. Optimality implies that
\[
\gamma\, \mu^* = \skal{B}{X^*}
\quad\Leftrightarrow\quad
\mu^* \skal{A}{X^*} = \skal{B}{X^*}
\quad\Leftrightarrow\quad
\skal{\mu^* A - B}{X^*} = 0.
\]

Note that in~\eqref{eq:dual} there is only one affine constraint, which
implies that there exists a rank-1 solution. Thus, we are aiming to
construct a rank-1 solution $X^* = x^* (x^*)\T \succeq 0$.  This solution
has to satisfy:
\[
0 = \skal{\mu^* A - B}{x^* (x^*)\T} = (x^*)\T (\mu^* A - B) x^*.
\]
Therefore, $x^*$ needs to be an eigenvector for the eigenvalue~$0$ of
$\mu^* A - B$ and thus $x^* \in \ker(\mu^* A - B)$. Moreover,
$\skal{A}{x^*(x^*)\T} = (x^*)\T A x^* = \gamma$ needs to hold.  Consider a
generator matrix $C \in \R^{n \times r}$ of the kernel of $\mu^* A - B$,
i.e., $\ker(\mu^* A - B) = \{C y \suchthat y \in \R^r\}$.  Then by scaling,
the problem reduces to the question whether there exists $y \in \R^r$ such
that
\[
y\T C\T A C y
\begin{cases}
  > 0 & \text{if } \gamma > 0,\\
  < 0 & \text{if } \gamma < 0.
\end{cases}
\]
Computing a maximal/minimal eigenvalue/eigenvector of $C\T A C$ will decide
this.

\paragraph{Case 2, $\boldsymbol{\mu^* = \ell}$:}

In this case, $\beta^* = 0$ has to hold.

If $\gamma \geq 0$, we can take
$(X^*, \alpha^*, \beta^*) = (0, \gamma, 0)$, which is feasible with
objective value $\gamma \ell = \gamma \mu^*$ as needed. Therefore, this
point is optimal.

If $\gamma < 0$, the conditions are
\[
\skal{B}{X^*} + \ell\, \alpha^* = \gamma\, \ell,\quad
\skal{A}{X^*} + \alpha^* = \gamma,\quad X^* \succeq 0,\; \alpha^* \geq 0.
\]
This yields
\[
\skal{B}{X^*} + \ell\, \alpha^* = \skal{A}{X^*}\, \ell + \alpha^*\, \ell
\quad\Leftrightarrow\quad
\skal{\ell\, A - B}{X^*} = 0.
\]
As above, we are looking for a rank-1 solution $(x^*)\T x^*$. Let
$C \in \R^{n \times r}$ be a generator matrix of $\ker(\ell\, A - B)$. Then
we need $y^* \in \R^r$ such that
\[
(y^*)\T C\T A C y^* < 0.
\]
If this is impossible, the problem is infeasible (e.g., if $A \succeq
0$). Otherwise, we can scale $x^* = C y^*$ and set $\alpha^* = 0$.

\paragraph{Case 3, $\boldsymbol{\mu^* = u}$:}

This case is symmetric to Case 2. If $\gamma < 0$, we can set
$(X^*, \alpha^*, \beta^*) = (0, 0, -\gamma)$.

If $\gamma \geq 0$, we can construct a rank-1 solution as above.

% -------------------------------------------------------------------------
\subsubsection{SDPs with All Fixed Variables}

Consider the general problem with fixed variables:
\[
  \inf\; \{ b\T y \suchthat \sum_{i=1}^m A^i\, y_i - A^0 \succeq 0,\; \ell \leq y \leq \ell\},
\]
and its dual:
\[
  \sup\; \{\skal{A^0}{X} + \ell\T \alpha - \ell\T \beta \suchthat \skal{A}{X} +
  \alpha_i - \beta_i = b_i\; \forall i \in [m],\; X \succeq 0,\; \alpha,\; \beta \geq 0\}.
\]
We can take
\[
X^* = 0,\quad
\alpha^*_i = \max\,\{0, b_i\}, \quad
\beta^*_i = -\min\,\{0, b_i\}.
\]
This solution is feasible for the dual and has objective value $\ell\T b$.

\begin{small}
   \bibliographystyle{plain}
   \bibliography{ConflictAnalysis.bib}
\end{small}

\end{document}

%%% Local Variables:
%%% mode: latex
%%% TeX-master: t
%%% End:
